%EINSTELLUNGEN
%Dokumenttyp und -format
%see KOMA-script manual <http://www.komascript.de>
\documentclass[	12pt, 
				a4paper, 
				BCOR=10mm, % Heftrandkorrektur
				DIV=12, 
				parskip=half, % Absatz mit halber Zeile Abstand, ansonsten nur eingerückt
				headings=small, % kleine Überschriften
				twoside, %oneside, %- je nach Umfang
				ngerman,
				bibliography=totoc,index=totoc, listof=totoc,
				numbers=noendperiod
				%,draft=true
				]{scrbook} %scrreprt bei kürzeren Arbeiten

%%%% Laden von Paketen %%%%

\usepackage{thetitlepage}

% Zeilenabstand 1.5
\usepackage{setspace}
\onehalfspacing %\singlespacing %\doublespacing
\KOMAoptions{DIV=last} %Seitenspiegel neu berechnen

% Zeilenabstand in Listen kleiner
\usepackage{enumitem}
\setlist{parsep=0.5\parskip} 

%automatische Silbentrennung
\usepackage[ngerman]{babel}% Das Beispieldokument ist in Deutsch,
% daher wird mit Hilfe des babel-Pakets
                % über Option ngerman auf deutsche Begriffe
                % und gleichzeitig Trennmuster nach den
                % aktuellen Rechtschreiberegeln umgeschaltet.
                % Alternativen und weitere Sprachen sind
                % verfügbar (siehe <http://ctan.org/pkg/babel>).
                


%automatische Übersetzung
\usepackage[ngerman]{translator}

%Schriftsatz
\usepackage[T1]{fontenc}
\usepackage{lmodern} %modernerer Schriftsatz
\KOMAoptions{DIV=last}%Seitenspiegel neu berechnen

%fuer deutsche Umlaute
\usepackage[utf8]{inputenc}

%deutsche Anfuehrungsstriche
\usepackage[babel,german=quotes]{csquotes}

% Literaturverwaltug
%\usepackage[style=debug]{biblatex}
\usepackage[
style=apa, 	% Zitierstil
isbn=false,	% ISBN nicht anzeigen, gleiches geht mit nahezu allen anderen Feldern
doi=false,
url=false, 
pagetracker=true,        % ebd. bei wiederholten Angaben (false=ausgeschaltet, page=Seite, spread=Doppelseite, true=automatisch)
%maxbibnames=50,       % maximale Namen, die im Literaturverzeichnis angezeigt werden (ich wollte alle)
%maxcitenames=3,        % maximale Namen, die im Text angezeigt werden, ab 4 wird u.a. nach den ersten Autor angezeigt
%autocite=inline,            % regelt Aussehen für \autocite (inline=\parancite)
block=space,               % kleiner horizontaler Platz zwischen den Feldern
backref=true,               % Seiten anzeigen, auf denen die Referenz vorkommt
backrefstyle=three+,    % fasst Seiten zusammen, z.B. S. 2f, 6ff, 7-10
%date=short,                  % Datumsformat
backend=biber            %Backend of choice, if bibtex is another option, but apa & bibtex may clash
]{biblatex}
\DeclareLanguageMapping{ngerman}{ngerman-apa}

 %Laden der Bibtex-Datei(en)
\addbibresource{literature.bib}


% Farbdefinitionen für Hyperref (s.u.) und Tabellen
\usepackage[table, svgnames]{xcolor}

%Url Anzeige
% Einstellungen für die Hyperlinks, die im PDF geklickt werden können
% Für den finalen Druck ist schwarzer Text besser, dann sollte die Option
% "colorlinks" gelöscht werden. Also so:
% \usepackage[unicode=true, breaklinks=true]{hyperref}
\usepackage[unicode=true, breaklinks=true, colorlinks=true] {hyperref}
\hypersetup{%
	citecolor=RoyalBlue,
	urlcolor=Crimson,
	linkcolor=ForestGreen
}
%Graphiken
\usepackage{graphicx}

%Mathebib
\usepackage{amsmath}
\usepackage{amssymb}

%Tabellen
\usepackage{tabularx}
%\usepackage{tabu}
%\usepackage[table]{xcolor}
\usepackage{multirow}
\usepackage{ctable} % needed for \cmidrule{}


%Farbe
\usepackage{color}

%Theorems
\usepackage{amsthm}

%Abkürzungen, Definitionen
% see http://en.wikibooks.org/wiki/LaTeX/Glossary
\usepackage[nomain,acronym,toc, translate=true]{glossaries} % nomain, if you define glossaries in a file, and you use \include{INP-00-glossary}
\makeglossaries
\makeindex

%%%% Einstellungen %%%%

%Gleitumgebungsbeschriftungen
%see KOMA-script manual
\renewcommand*{\captionformat}{.\ }
\renewcommand*{\tableformat}{\tablename~\thetable}
\addtokomafont{caption}{}
\setlength\abovecaptionskip{\lineskip}
\setcapindent{0em}
\setkomafont{captionlabel}{\sffamily\bfseries}
\KOMAoptions{
captions=tableheading, 	%Tabellen als Überschrift
captions=figuresignature,	%Abbildungen als Unterschrift
captions=leftbeside,
captions=nooneline
}

% Zuweisung von Schriften
% see KOMA-script manual
\setkomafont{subject}{\normalfont\large}

% Format des Anhangs: Anhang A. name, Inhalte nicht im Inhaltsverzeichnis
% see KOMA-script manual
\newcommand*{\appendixmore}{% 
\renewcommand*{\chapterformat}{%
\appendixname~\thechapter.\enskip}% 
\renewcommand*{\chaptermarkformat}{%
\appendixname~\thechapter.\enskip}
\renewcommand*{\othersectionlevelsformat}[3]{%
##3.\enskip}%
\renewcommand*{\sectionmarkformat}{%
\thesection.\enskip}
\addcontentsline{toc}{chapter}{\appendixname}% 
\addtocontents{toc}{\protect\value{tocdepth}=-1}% 
}

\theoremstyle{plain}% default
\newtheorem{thm}{Theorem}[section]
\newtheorem{lem}[thm]{Lemma}
\newtheorem{prop}[thm]{Proposition}
\newtheorem*{cor}{Corollary}
\newtheorem*{KL}{Klein’s Lemma}
\theoremstyle{definition}
\newtheorem{alg}[thm]{Algorithmus}
\newtheorem{defn}{Definition}[section]
\newtheorem{conj}{Conjecture}[section]
\newtheorem{exmp}{Example}[section]
\theoremstyle{remark}
\newtheorem*{rem}{Remark}
\newtheorem*{note}{Note}
\newtheorem{case}{Case}

%%%% Eigene Definitionen %%%%

\makeatletter
%environmentdefinitionen hierhinein
\makeatother

%eigene Definitionen
%\newcommand{vec}[1]{\ensuremath\textbf{#1}}   
\newcommand{\vc}[1]{\ensuremath\textbf{#1}} 
\newcommand{\qt}[1]{\ensuremath{\dot{#1}}} 
\newcommand{\mt}[1]{\ensuremath\textbf{\uppercase{#1}}} 
\newcommand{\pt}[1]{\ensuremath{\textbf{\textit{\uppercase{#1}}}}} 
\newcommand{\kor}[1]{\textcolor{red}{#1}}

\newcommand{\abbpos}{50pt}

%Abkürzungen
% see http://en.wikibooks.org/wiki/LaTeX/Glossary
%new glossary term
\newglossaryentry{sample}{name={sample}, description={a sample entry}}

%new acronym
\newacronym[\glsshortpluralkey=cas,\glslongpluralkey=contrived acronyms]{aca}{aca}{a contrived acronym}%
\newacronym{abk}{Abk.}{Abkürzung}%
\newacronym{http}{HTTP}{hypertext transfer protokol}


% DOKUMENT ANFANG
\begin{document}

%%%% Titelseite Definitionen %%%%

%Header PsyErg
%\def\tabularxcolumn#1{m{#1}}
% \renewcommand{\thetitlehead}{
%\begin{tabularx}{\textwidth}{l X m{6.5cm}}
%\raisebox{-.5\height}{
%\includegraphics[height=1.8cm]{figures/unilogo4cohne.jpg}
%} & &\small{Philosophische Fakultät II\newline
%Institut für Mensch-Computer-Medien\newline{Psychologische Ergonomie}}\\   
%\end{tabularx} }

%Header HCI
\def\tabularxcolumn#1{m{#1}}
 \renewcommand{\thetitlehead}{
\begin{tabularx}{\textwidth}{l X m{6.745cm} r}
\raisebox{-.5\height}{
\includegraphics[height=1.8cm]{figures/unilogo4cohne.jpg}
} & &\small{Fakultät für Mathematik und Informatik\newline
Institut für Informatik\newline{Mensch-Computer Interaktion}}&
\raisebox{-.5\height}{
\includegraphics[height=1.8cm]{figures/hci-logo-red.pdf}
}\\ \end{tabularx} }


%Art der Arbeit (Seminararbeit/Hausarbeit zum Seminar "Seminarname" "Semester", Bachelorarbeit,...)
\renewcommand{\artderarbeit}{ \usekomafont{subject}\textbf{Projektarbeit}  \vspace{0.5cm} \\ \normalfont zum Kurs\\ Grundlagen der Mensch-Computer-Systeme \\ im Studiengang Master Informatik\\ an der Universität Würzburg}

%Titel der Arbeit
\renewcommand{\thetitle}{Cognitive Walkthrough zum Bestellprozess des Onlineshops www.alternate.de}

% ggf. Untertitel der Arbeit
%\renewcommand{\thesubtitle}{ Untertitel}

%Autor(en)
\renewcommand{\theauthor}{vorgelegt von\\ Bastian Hedenkamp - Matrikelnummer: 1949204\\ Michael Gabler - Matrikelnummer: 2349277}

%Datum (Bei Thesis Abgabedatum)
\renewcommand{\thedate}{am \today}

%Betreuer/ Kursleiter
\renewcommand{\betreuer}{Betreuer/Prüfer:\\ Prof. Marc Erich Latoschik, Informatik IX, Universität Würzburg}


%%%% Zusammenfassung %%%%

% obligatorisch bei Abschlussarbeiten
%\renewcommand{\theabstractfirst}{deutsche Zusammenfassung - Lorem ipsum dolor sit amet, consetetur sadipscing elitr, sed diam nonumy eirmod tempor invidunt ut labore et dolore magna aliquyam erat, sed diam voluptua. At vero eos et accusam et justo duo dolores et ea rebum. Stet clita kasd gubergren, no sea takimata sanctus est Lorem ipsum dolor sit amet. Lorem ipsum dolor sit amet, consetetur sadipscing elitr, sed diam nonumy eirmod tempor invidunt ut labore et dolore magna aliquyam erat, sed diam voluptua. At vero eos et accusam et justo duo dolores et ea rebum. Stet clita kasd gubergren, no sea takimata sanctus est Lorem ipsum dolor sit amet.   

%Duis autem vel eum iriure dolor in hendrerit in vulputate velit esse molestie consequat, vel illum dolore eu feugiat nulla facilisis at vero eros et accumsan et iusto odio dignissim qui blandit praesent luptatum zzril delenit augue duis dolore te feugait nulla facilisi. Lorem ipsum dolor sit amet}

%%%% Titelei %%%%

\frontmatter%bei scrreprt auskommentieren
\begin{spacing}{0}

% Titelseite
\printthetitlepage
\end{spacing}

% Zusammenfassung (obligatorisch bei Abschlussarbeiten)
%\printtheabstract{Zusammenfassung} 	%\printtheabstract{titlefirst} nur %\abstractfirst mit Titel 'fitlefirst'

\begin{spacing}{0}

% Inhaltsverzeichnis
\tableofcontents 	

% Abbildungsverzeichnis  (optional)
%\listoffigures 		

% Tabellenverzeichnis (optional)
%\listoftables		

% Abkürzungsverzeichnis
% You have to run:
% makeindex -s "$bfname".ist -t "$bfname".alg -o "$bfname".acr "$bfname".acn
% makeindex -s "$bfname".ist -t "$bfname".glg -o "$bfname".gls "$bfname".glo
\printglossary[type=\acronymtype] 

\end{spacing}

%%%% Inhalte %%%%

\mainmatter %bei scrreprt auskommentieren

\chapter{Einleitung}
Im Folgenden wird die Webseite von Alternate evaluiert. Die Webseite ist unter der URL \url{https://www.alternate.de} erreichbar. Bei der Alternate-Webseite handelt es sich um einen klassischen Onlineshop der hauptsächlich Elektronik vertreibt. Zu dem Sortiment gehören unter anderem Hardware-Komponenten für den Eigenbau von PCs, komplette PCs und Laptops, Smartphones, Unterhaltungselektronik wie Fernseher, Haushaltsgeräte, elektrische Werkzeuge, sowie Spielzeug und Elektrofahrräder. Das Angebot richtet sich daher generell an alle potentiellen Kunden, die online einkaufen. Da die Zielgruppe keinerlei Beschränkungen unterliegt, sollte die Webseite sowohl von neuen als auch von bestehenden Kunden gleichermaßen einfach bedienbar sein. Insbesondere sollten sich bei dem Auftritt des Onlineshops auch wenig erfahrene Nutzer ohne Hilfestellung selbständig zurechtfinden können. Hauptsächlich werden die Nutzer der Seite technikinteressiert sein und mit der Funktionsweise von Onlineshops im Allgemeinen vertraut sein.
Die Evaluation wird mittels der Cognitive Walkthrough Methode durchgeführt. Bei der Durchführung der Methode werden Personas erstellt aus deren Sicht verschiedene Aufgaben mit konkretem Ziel auf der Webseite durchgeführt werden. Hierbei können die Experten schwächen im Design der grafischen Benutzeroberfläche finden und anschließend die Bedienprobleme durch neue Designs beheben.

\chapter{Methodik}
Die Methode Cognitive Walkthrough wird dazu verwendet um das Design einer Benutzeroberfläche zu beurteilen. Die Methode kann sowohl von den Entwicklern des Designs, als auch von Experten durchgeführt werden. Diese Methode hat den Vorteil, dass für die Durchführung keine Versuchspersonen benötigt werden. Der Cognitive Walkthrough kann sowohl zum Bewerten von Prototypen, als auch von fertigen Systemen verwendet werden. Vor allem beim Testen eines Prototypen können Bedienprobleme rechtzeitig erkannt und kostengünstig verbessert werden. Bei einem fertigen System hingegen sind späte Änderungen der Benutzeroberfläche meist viel aufwändiger und teurer. Zur Vorbereitung werden potentielle Nutzergruppen erstellt, so wie deren Vorwissen ermittelt. Für diese Nutzergruppen werden Aufgaben mit konkreten Zielen erstellt, die diese mithilfe des zu evaluierenden Systems erreichen sollen. Bei der Durchführung der Methode führt der Entwickler bzw. der Experte folgende Schritte durch: Zuerst wird die Benutzeroberfläche aus der Sichtweise eines Nutzers aus der definierten Gruppe betrachtet und nach verfügbaren Aktionen durchsucht. Danach wird die Aktion gewählt und ausgeführt die aus Sicht des Nutzers am wahrscheinlichsten zum Ziel führt. Nach der Aktion wird das Interface überprüft ob das System Feedback gibt oder es Hinweise für Fortschritt gibt. Diese Schritte werden wiederholt bis entweder das Ziel erreicht wird oder der Test abgebrochen werden muss. Während dieser Schritte achtet der Entwickler bzw. der Experte darauf ob die richtigen Aktionen gewählt wurden und ob Probleme auftreten. Alle gefundenen Bedienprobleme können durch eine Verbesserung des Designs der Benutzeroberfläche behoben werden. Mit dem neuen Design kann die Methode erneut ausgeführt werden und das Design weiter verbessert werden.

\chapter{Durchführung}
\section{Vorbereitung}
Zur Vorbereitung werden zuerst die Personengruppen ermittelt, die voraussichtlich Nutzer der Webseite alternate.de sind. Da der Onlineshop hauptsächlich technische Produkte aus den Bereichen Hardware, Software und Komplettsysteme anbietet, wird davon ausgegangen, dass die primäre Zielgruppe technikinteressierte Personen sind, welche die Produkte für den Privatgebrauch erwerben möchten. Durch das Interesse für Technik sollte diese Nutzergruppe bereits mit gängigen Bedienkonzepten von Webseiten und Onlineshops vertraut sein, auch ein sicherer Umgang mit PC und Smartphone wird angenommen. Das Alter der Nutzer bewegt sich voraussichtlich zwischen 16 und 60 Jahren und könnte Einfluss auf das Endgerät haben, mit dem der Onlineauftritt besucht wird. Dabei wird angenommen, dass jüngere Nutzer den Onlineshop eher mit mobilen Geräten wie Smartphone oder Tablet besuchen, ältere Nutzer vermutlich eher zum Laptop oder PC greifen.\\
Neben der primären ergeben sich weitere potentielle Nutzergruppen. Da der Shop außer den bereits genannten Produkten auch solche aus den Kategorien Smart Home, Multimedia oder Outdoor, wie Campingzubehör oder Werkzeug anbietet, gehören auch Personen zum Kundenkreis, die eher weniger technisch versiert sind, jedoch trotzdem eine gewisse Vorerfahrung in der Bedienung von Webseiten besitzen. Dazu zählen Hobbygärtner oder Rentner, die nach neuen Gartengeräten suchen, Hobbybastler, die ihr Haus mit Smart Home Geräten ausstatten möchten oder (Hobby-)Fotografen, die sich über neue Kameras informieren bzw. diese erwerben wollen.\\
Als dritte Nutzergruppe lassen sich gewerbliche Kunden ausmachen, die Hardware oder andere Produkte für deren Endkunden oder den eigenen Bedarf benötigen, wie z.B. Netzwerktechniker. Dieser Gruppe wird ebenfalls eine breite Vorerfahrung mit anderen Webseiten und Onlineshops besitzen und sich daher mit standardisierten Bedienkonzepten zurecht finden. Als Endgeräte werden von diesen Personen voraussichtlich die von deren Arbeitgeber bereitgestellten Laptops oder PCs verwendet.\\
Bei den hier zu analysierenden Szenarien wurde sich hauptsächlich auf die primäre Nutzergruppe konzentriert, da der Onlineshop in erster Linie für diese Personen optimiert sein sollte. Im Folgenden soll daher die Produktsuche und der Bestellprozess anhand zweier Produkte untersucht werden. Dabei wird einmal davon ausgegangen, dass der Kunde seine Bestellung an einem PC tätigt und bereits ein Kundenkonto besitzt. Dies könnte somit ein Stammkunde und/oder ein Geschäftskunde sein. Im zweiten Szenario gibt ein Neukunde eine Bestellung auf, der kein Kundenkonto erstellen will, also einmalig als Gast bestellen möchte und hierfür sein Smartphone nutzt. Damit kann es sich hierbei um eine Privatperson jüngeren Alters handeln, beispielsweise ein Student der Informatik. In beiden Szenarien wird von einer grundlegenden Kenntnis der gängigen Bedienkonzepte von Webseiten ausgegangen, wie dies für die primäre Nutzergruppe beschrieben wurde.

\section{Analyse}
\subsection{Bestellprozess angemeldet am PC}
Die Aufgabe in dieser Evaluation ist es das günstigste \textit{8 GB DDR4 RAM} Modul auf der Alternate-Webseite zu finden. Der Nutzer benutzt hierfür die Desktop-Variante der Website und besitzt bereits ein Benutzerkonto.
\paragraph{Navigation über Suche}
Das Suchfeld ist mittig in den Header der Webseite integriert und wandert beim Scrollen mit. Durch den Platzhalter \glqq Suchbegriff eingeben...\grqq{} wird deutlich, dass es sich hierbei um die Suche handelt. Der Button zum Starten der Suche rechts daneben ist grau und beinhaltet ein weißes Dreieck, welches nach rechts zeigt.
Wenn der Nutzer das Suchfeld im Kopf der Seite entdeckt hat gibt er hier die Suchbegriffe \glqq 8 gb ddr4\grqq{} ein. Daraufhin erhält man circa 1000 Ergebnisse die nach Beliebtheit sortiert sind. Auf der linken Seite befindet sich eine sehr lange Liste mit Filtern. In der Mitte sind 40 Suchergebnisse in einer Liste untereinander zu sehen. Die Suchergebnisse enthalten nicht nur die gesuchten Module, sondern auch größere und kleinere Kits, sowie Laptops in denen DDR4 Speicher verbaut ist und Mainboards welche DDR4 Speicher unterstützen.
Um das günstigste Modul zu erhalten wählt man die Sortierung nach Preis. Daraufhin erhält man passende Module zur Aufgabenstellung und hat nun die Möglichkeit ein solches Modul zu kaufen.
\paragraph{Navigation über Menü}
Angenommen der Nutzer ist mit den Suchergebnissen nicht zufrieden oder er möchte die Suche nicht benutzten. In diesem Fall würde der Nutzer das Menü nutzen und dieses manuell nach dem Begriff \textit{RAM} oder \textit{Arbeitsspeicher} durchsuchen. Das Menü umfasst 20 Einträge die Aufklappen wenn der Mauszeiger auf einem Eintrag stehen bleibt. Der vielversprechendste Menüeintag ist \textit{Hardware} und unter diesem gibt es 96 untergeordnete Einträge, welche auf acht Spalten verteilt sind. Nach etwas Suche findet man den Eintrag \textit{Arbeitsspeicher}. Links auf der Seite sind die Unterkategorien von \textit{Hardware} zu sehen, sowie die Unterkategorien von \textit{Arbeitsspeicher}. Nun wählt man links in der Liste \textit{DDR4 RAM} aus um die Auswahl zu verfeinern. In der Mitte sind wie bei der Suche 40 Produkte aufgelistet. Über dieser Liste können Filter gewählt werden. Hier wählt man die Filter \glqq Speicherkapazität=8.192 MB\grqq{} und \glqq Anzahl der Speichermodule = 1\grqq{}. Danach lässt man die Produkte nach dem Preis sortieren. An dieser Stelle fällt auf, dass man etwas andere Ergebnisse erhält als bei der Suche. Das ist darauf zurückzuführen, dass die suche nur die Namen der Produkte berücksichtigt. Produkteigenschaften die in dem Namen nicht erwähnt werden können über die Suche daher nicht gefunden werden.
\paragraph{Bestellprozess}
Aus der Liste mit den Produkten kann nun ein Produkt ausgewählt werden. Dort klickt man auf den \glqq In den Warenkorb\grqq{}-Button. Daraufhin erhält man eine Bestätigung, dass das Produkt dem Warenkorb hinzugefügt wurde. Dort klickt man auf den Button \glqq Zur Kasse\grqq{}. Auf der darauf folgenden Seite loggt man sich mit seiner E-Mail-Adresse oder Kundennummer und Passwort ein. Zuletzt kann man noch die Bestellung und die eigenen Daten prüfen sowie Zahlungsart wählen und den Kauf abschließen.

\subsection{Bestellprozess als Gast mit Smartphone}
Dieses Szenario beschreibt einige Schritte, die ein Nutzer durchlaufen könnte, wenn er mit einem Smartphone eine Grafikkarte mit dem Chip \textit{NVIDIA GTX 1070} bei alternate.de bestellen möchte. Der Nutzer ist Neukunde und somit nicht mit der Webseite vertraut, kennt jedoch die Bedienung anderer Onlineshops. Er möchte zudem kein Kundenkonto erstellen, sondern lediglich einmalig als Gast bestellen.\\
Der Nutzer startet auf der mobilen Version von alternate, wenn alternate.de auf dem Smartphone geöffnet wird. Dort befindet sich zuerst ein großes Werbebanner und darunter eine Auswahl der beliebtesten Kategorien mit Bild. Hier könnte der Nutzer die Kategorie \textit{Hardware} wählen, da auf der Kachel bereits eine Grafikkarte abgebildet ist. Alternativ mögliche Kategorien wären \textit{Gaming} oder \textit{PC}. Außerdem befindet sich rechts oben ein Lupensymbol, welches auf eine Suchfunktion hindeutet. Auf der linken Seite kann zudem über ein Symbol mit drei Balken eine Menüleiste ausgeklappt werden, welche allerdings die gleichen Kategorien wie auf der Startseite enthält. Damit könnten auch die oben genannten Kategorien ausgewählt werden.

\paragraph{Navigation über Kategorien} Bei der Auswahl einer der genannten Kategorien wird der Nutzer feststellen, dass lediglich die Kategorie \textit{Hardware} auf eine Seite weiterleitet, die sinnvolle Unterkategorien enthält, um eine Grafikkarte zu finden. Die Kategorien \textit{Gaming} und \textit{PC} führen daher nicht zum Ziel. Über die Kategorie \textit{Hardware} kann der Nutzer durch die Unterkategorien \textit{PC-Komponenten}, \textit{Grafikkarten} zu einer Produktübersicht navigieren, wobei bei der Auswahl der nächsten Unterkategorie nach unten gescrollt werden muss, da der obere Bildschirmbereich von einem Werbebanner eingenommen wird. Die Produktübersicht lässt sich über die Filteroption \textit{NIVIDIA Grafikkarten} weiter einschränken. Alternativ kann im Menü \textit{Filtern} direkt der Grafikchip ausgewählt werden. Auffällig ist hier die Vielzahl der Filtermöglichkeiten, die nicht immer eine eindeutige Beschriftung sowie Auswahl aufweisen, z.B. kann für den Filter \textit{Nur sofort lieferbar} nur \textit{Ändern} oder \textit{ja} ausgewählt werden. Die nun gefilterte Produktübersicht enthält dann alle relevanten Produkte mit Preisangabe und aktueller Verfügbarkeit. Durch Anklicken eines Produkts kann man es auf der sich dann öffnenden Produktseite in den Warenkorb legen. Allerdings wurden bei der Durchführung des Cognitive Walkthroughs auch Produkte entdeckt, die in der Übersicht als lieferbar angezeigt wurden, es auf der Produktseite dann aber nicht möglich war, sie in den Warenkorb zu legen.

\paragraph{Navigation über Suche} Um das passende Produkt zu finden, kann der Nutzer auch direkt auf der Startseite die Suchfunktion verwenden. Durch Anklicken des Lupensymbols öffnet sich eine Eingabeleiste. Wird dort nach dem Suchbegriff \textit{NVIDIA} gesucht, erscheint ebenfalls wie bei der Kategorieauswahl eine Produktübersicht in der sich allerdings hauptsächlich PC-Monitore wiederfinden. Erst die Suche nach \textit{Grafikkarte} oder \textit{GTX 1070} resultiert in einer sinnvollen Anzeige diverser Produkte, wobei bei letzterem Suchbegriff neben einzelnen Grafikkarten auch PC-Komplettsysteme mit einer verbauten Grafikkarte vom angegebenen Typ angezeigt werden. Die Suchergebnisse können in beiden Fällen noch über die oben beschriebenen Filter verfeinert werden, die hier allerdings zahlreicher sind und auch irrelevante Kriterien wie z.B. \textit{Displaytyp} enthalten. Aus der Produktliste kann wie bei der Navigation über Kategorien nun das passende Produkt gewählt und auf der Produktseite in den Einkaufswagen gelegt werden.

\paragraph{Bestellprozess} Hat der Nutzer ein Produkt zum Warenkorb hinzugefügt, findet er sich auf einer Bestätigungsseite, von der aus er zum Warenkorb gelangt. Dabei zeigt auch das Warenkorbsymbol oben rechts die Anzahl der im Warenkorb vorhandenen Produkte an. Im Warenkorb selbst sieht der Nutzer eine Übersicht der enthaltenen Produkte und deren Preis. Er kann nun direkt über die Zahlungsanbieter \textit{PayPal} oder \textit{masterpass} bestellen und wird dann zur Seite des jeweiligen Anbieters weitergeleitet, ohne ein Kundenkonto zu erstellen. Alternativ gelangt der Kunde über den Button \textit{Zur Kasse} zu einem Anmeldeformular für ein bestehendes Kundenkonto oder der Möglichkeit ein neues zu erstellen. Die Bestellung als Gast ist nur möglich, wenn der Nutzer auf den Link \textit{zur klassischen Ansicht} klickt, wodurch er auf die Startseite des Shops weitergeleitet wird, die nicht für Mobilgeräte optimiert ist. Hier muss auch die Produktauswahl erneut erfolgen, da der Warenkorb nicht übernommen wird. Erstellt der Benutzer über die mobile Seite ein Kundenkonto, muss er sich im Anschluss daran einloggen und kann dann bestellen.

\section{Ergebnisse}

\paragraph{Bestellprozess angemeldet am PC}
Das Suchfeld kann übersehen werden und sollte daher noch präsenter gestaltet werden.\\
Das Menü mit den Kategorien hat zu viele Einträge und die Menge der untergeordneten Einträge ist viel zu groß. Das hat zur Folge, dass der Nutzer die Eintrage manuell durchsuchen muss, indem er diese sequentiell durchliest, bis er gefunden hat wonach er sucht.\\
Bei der Suche ist die Spalte mit den Filtern zu lang und zu überladen. Filter mit Speichermengen in MB, wie beim Arbeitsspeicher, sind zu lang und wirken altmodisch.\\
Die Suche beschränkt sich auf den Namen des Produktes, sodass man nicht immer das findet wonach man sucht, obwohl ein passendes Produkt im Shop vorhanden ist.

\paragraph{Bestellprozess als Gast mit Smartphone} Bei der Bestellung per Smartphone wird auf der Startseite bereits die Vielzahl der Kategorien zum Problem. Hier kann der Nutzer leicht die falsche auswählen, weil sich ein gewünschtes Produkt hinter mehreren Hauptkategorien verbergen könnte. Da die Startseite bereits die wichtigsten Kategorien auflistet, ist das ausklappbare Menü redundant und wird vermutlich nicht immer eindeutig gefunden, da das Icon mit den drei Balken abgeschnitten wurde. Werden anschließend die Unterkategorien ausgewählt, stellt sich das auf jeder Seite oben angebrachte Werbebanner als störend heraus, da die für den Nutzer interessanten Inhalte nach unten rücken und somit zum Teil erst nach dem Scrollen sichtbar werden. Die Suchfunktion wird hingegen vermutlich gut gefunden, liefert allerdings nicht immer die erwarteten Suchergebnisse. Auch die Filterfunktionen sind zwar hilfreich, allerdings erscheinen diese teils nicht kontextbezogen und die Filterwerte teils unpassend. Zudem hebt sich die Schrift durch zu niedrigen Kontrast wenig vom Hintergrund ab, was für Personen mit Seheinschränkungen gerade auf Smartphones Schwierigkeiten verursachen kann. Das Abschließen des Bestellprozesses über einen Drittanbieter wie \textit{Paypal} oder \textit{masterpass} ist gut umgesetzt und ermöglicht eine unkomplizierte Bestellung als Gast. Ist der Nutzer allerdings nicht Kunde bei einem der zwei Anbieter, hat er praktisch keine Möglichkeit, als Gast zu bestellen, da das Umschalten auf die PC-Ansicht und das erneute Auswählen der Produkte unzumutbar ist.

\section{Diskussion}

Um das Suchfeld in der Desktop-Version der Seite noch präsenter zu machen kann das Suchfeld vergrößert werden und das Symbol von dem Button durch eine Lupe ersetzt werden. Dadurch wäre die Symbolik deutlicher als mit dem Dreieck.
Die Suchergebnisse könnten verbessert werden, indem weitere Produkteigenschaften in der Suche berücksichtigt werden würden, anstatt nur dem Produktnamen. Alternativ könnten Produkten auch Tags zugewiesen werden, sodass wenn ein Tag mit deinem Suchwort übereinstimmt alle Produkte mit diesem Tag angezeigt werden.\\
Das überladene Menü mit den Kategorien sollte durch die Reduzierung der Einträge verkleinert werden. Es wäre ratsam sowohl für die Haupteinträge als auch für die untergeordneten Einträge die 7+-2 Regel zu befolgen um dem Nutzer die manuelle Suche zu ersparen.\\
Die Filter sollten übersichtlicher gestaltet werden. Wenn viele Filter vorhanden sind sollten diese zugeklappt sein, mit der Möglichkeit diese aufzuklappen und zu benutzen. Filteroptionen sollten nicht angezeigt werden wenn kein Kontext erkennbar ist. Für den Fall, dass der Filter binär ist, wie zum Beispiel \glqq Nur verfügbare anzeigen\grqq{} , wäre auch bei der mobilen Variante eine Checkbox zum aktivieren des Filters sinnvoll.
Filter mit der Einheit MB sollten modernisiert werden und stattdessen in GB angegeben werden.\\
Das Doppelte Menü in der mobilen Variante der Website sollte durch ein zentrales Menü ersetzt werden. Auf der Mobilen Seite sollte mit Werbebannern sparsam umgegangen werden da hier weniger Platz verfügbar ist. Zum Beispiel könnte ein einzelner Werbebanner unter der Kategorieauswahl platziert werden.
Für einen konsistenten Auftritt wäre es gut wenn Alternate eine Website hätte die responsive ist. Dadurch wären die Unterschiede im Design und in der Funktionalität zwischen der Desktop-Variante und der mobilen Variante der Website nicht so groß. Für die mobile Variante wäre ein erhöhter Kontrast zwischen Hintergrund und Schrift angebracht, da es auch mobile Geräte mit kontrastarmen Displays gibt und eine Benutzung unter freiem Himmel mit Sonnenschein möglich sein sollte.\\
Um die Anzahl der Klicks zu reduzieren wäre es gut, wenn es schon bei der Produktliste eine Möglichkeit gäbe ein Produkt direkt in den Einkaufswagen zu legen.

\chapter{Zusammenfassung}
Durch die Methode des Cognitive Walkthrough konnten für diesen Onlineshop anhand verschiedener Szenarien und im Bezug auf die potentiellen Nutzergruppen einige Usabilityprobleme identifiziert werden. Kritisch ist beispielsweise, dass auf Mobilgeräten Funktionen nicht vorhanden sind, die dort durchaus Sinn machen würden, aber nur über die PC-Webseite zugänglich sind. Andere Probleme erschweren für den Nutzer die Bedienung und verlängern die Zeit, sein Ziel zu erreichen, machen es aber nicht unmöglich, eine Bestellung zu tätigen. Durch einfache Redesigns und der eventuellen Konsolidierung des mobilen und des PC-Auftritts, können viele der Probleme behoben werden und ein Mehrwert sowohl für Nutzer als auch Betreiber des Onlineshops geschaffen werden.

%%%% Anhänge %%%%

\newpage
 \printbibliography	% Literaturvezeichnis (obligatorisch)
%\printbibheading
%\printbibliography[nottype=online,heading=subbibliography,
%                   title={Printed Sources}]
%\printbibliography[type=online,heading=subbibliography,
%                   title={Online Sources}]

\appendix 				% Anhang (optional) Überschriften beginnen mit Anhang ...

\newpage
\chapter{Authorenverzeichnis}
Gemeinsam: Einleitung, Zusammenfassung\\
Bastian Hedenkamp: Bestellprozess angemeldet am PC, Ergebnisse, Diskussion \\
Michael Gabler: Vorbereitung, Bestellprozess als Gast mit Smartphone, Ergebnisse, Prototyp 


\end{document}
% DOKUMENT ENDEU