%EINSTELLUNGEN
%Dokumenttyp und -format
%see KOMA-script manual <http://www.komascript.de>
\documentclass[	12pt, 
				a4paper, 
				BCOR=10mm, % Heftrandkorrektur
				DIV=12, 
				parskip=half, % Absatz mit halber Zeile Abstand, ansonsten nur eingerückt
				headings=small, % kleine Überschriften
				twoside, %oneside, %- je nach Umfang
				ngerman,
				bibliography=totoc,index=totoc, listof=totoc,
				numbers=noendperiod
				%,draft=true
				]{scrbook} %scrreprt bei kürzeren Arbeiten

%%%% Laden von Paketen %%%%

\usepackage{thetitlepage}

% Zeilenabstand 1.5
\usepackage{setspace}
\onehalfspacing %\singlespacing %\doublespacing
\KOMAoptions{DIV=last} %Seitenspiegel neu berechnen

% Zeilenabstand in Listen kleiner
\usepackage{enumitem}
\setlist{parsep=0.5\parskip} 

%automatische Silbentrennung
\usepackage[ngerman]{babel}% Das Beispieldokument ist in Deutsch,
% daher wird mit Hilfe des babel-Pakets
                % über Option ngerman auf deutsche Begriffe
                % und gleichzeitig Trennmuster nach den
                % aktuellen Rechtschreiberegeln umgeschaltet.
                % Alternativen und weitere Sprachen sind
                % verfügbar (siehe <http://ctan.org/pkg/babel>).
                


%automatische Übersetzung
\usepackage[ngerman]{translator}

%Schriftsatz
\usepackage[T1]{fontenc}
\usepackage{lmodern} %modernerer Schriftsatz
\KOMAoptions{DIV=last}%Seitenspiegel neu berechnen

%fuer deutsche Umlaute
\usepackage[utf8]{inputenc}

%deutsche Anfuehrungsstriche
\usepackage[babel,german=quotes]{csquotes}

% Literaturverwaltug
%\usepackage[style=debug]{biblatex}
\usepackage[
style=apa, 	% Zitierstil
isbn=false,	% ISBN nicht anzeigen, gleiches geht mit nahezu allen anderen Feldern
doi=false,
url=false, 
pagetracker=true,        % ebd. bei wiederholten Angaben (false=ausgeschaltet, page=Seite, spread=Doppelseite, true=automatisch)
%maxbibnames=50,       % maximale Namen, die im Literaturverzeichnis angezeigt werden (ich wollte alle)
%maxcitenames=3,        % maximale Namen, die im Text angezeigt werden, ab 4 wird u.a. nach den ersten Autor angezeigt
%autocite=inline,            % regelt Aussehen für \autocite (inline=\parancite)
block=space,               % kleiner horizontaler Platz zwischen den Feldern
backref=true,               % Seiten anzeigen, auf denen die Referenz vorkommt
backrefstyle=three+,    % fasst Seiten zusammen, z.B. S. 2f, 6ff, 7-10
%date=short,                  % Datumsformat
backend=biber            %Backend of choice, if bibtex is another option, but apa & bibtex may clash
]{biblatex}
\DeclareLanguageMapping{ngerman}{ngerman-apa}

 %Laden der Bibtex-Datei(en)
\addbibresource{literature.bib}


% Farbdefinitionen für Hyperref (s.u.) und Tabellen
\usepackage[table, svgnames]{xcolor}

%Url Anzeige
% Einstellungen für die Hyperlinks, die im PDF geklickt werden können
% Für den finalen Druck ist schwarzer Text besser, dann sollte die Option
% "colorlinks" gelöscht werden. Also so:
% \usepackage[unicode=true, breaklinks=true]{hyperref}
\usepackage[unicode=true, breaklinks=true, colorlinks=true] {hyperref}
\hypersetup{%
	citecolor=RoyalBlue,
	urlcolor=Crimson,
	linkcolor=ForestGreen
}
%Graphiken
\usepackage{graphicx}

%Mathebib
\usepackage{amsmath}
\usepackage{amssymb}

%Tabellen
\usepackage{tabularx}
%\usepackage{tabu}
%\usepackage[table]{xcolor}
\usepackage{multirow}
\usepackage{ctable} % needed for \cmidrule{}


%Farbe
\usepackage{color}

%Theorems
\usepackage{amsthm}

%Abkürzungen, Definitionen
% see http://en.wikibooks.org/wiki/LaTeX/Glossary
\usepackage[nomain,acronym,toc, translate=true]{glossaries} % nomain, if you define glossaries in a file, and you use \include{INP-00-glossary}
\makeglossaries
\makeindex

%%%% Einstellungen %%%%

%Gleitumgebungsbeschriftungen
%see KOMA-script manual
\renewcommand*{\captionformat}{.\ }
\renewcommand*{\tableformat}{\tablename~\thetable}
\addtokomafont{caption}{}
\setlength\abovecaptionskip{\lineskip}
\setcapindent{0em}
\setkomafont{captionlabel}{\sffamily\bfseries}
\KOMAoptions{
captions=tableheading, 	%Tabellen als Überschrift
captions=figuresignature,	%Abbildungen als Unterschrift
captions=leftbeside,
captions=nooneline
}

% Zuweisung von Schriften
% see KOMA-script manual
\setkomafont{subject}{\normalfont\large}

% Format des Anhangs: Anhang A. name, Inhalte nicht im Inhaltsverzeichnis
% see KOMA-script manual
\newcommand*{\appendixmore}{% 
\renewcommand*{\chapterformat}{%
\appendixname~\thechapter.\enskip}% 
\renewcommand*{\chaptermarkformat}{%
\appendixname~\thechapter.\enskip}
\renewcommand*{\othersectionlevelsformat}[3]{%
##3.\enskip}%
\renewcommand*{\sectionmarkformat}{%
\thesection.\enskip}
\addcontentsline{toc}{chapter}{\appendixname}% 
\addtocontents{toc}{\protect\value{tocdepth}=-1}% 
}

\theoremstyle{plain}% default
\newtheorem{thm}{Theorem}[section]
\newtheorem{lem}[thm]{Lemma}
\newtheorem{prop}[thm]{Proposition}
\newtheorem*{cor}{Corollary}
\newtheorem*{KL}{Klein’s Lemma}
\theoremstyle{definition}
\newtheorem{alg}[thm]{Algorithmus}
\newtheorem{defn}{Definition}[section]
\newtheorem{conj}{Conjecture}[section]
\newtheorem{exmp}{Example}[section]
\theoremstyle{remark}
\newtheorem*{rem}{Remark}
\newtheorem*{note}{Note}
\newtheorem{case}{Case}

%%%% Eigene Definitionen %%%%

\makeatletter
%environmentdefinitionen hierhinein
\makeatother

%eigene Definitionen
%\newcommand{vec}[1]{\ensuremath\textbf{#1}}   
\newcommand{\vc}[1]{\ensuremath\textbf{#1}} 
\newcommand{\qt}[1]{\ensuremath{\dot{#1}}} 
\newcommand{\mt}[1]{\ensuremath\textbf{\uppercase{#1}}} 
\newcommand{\pt}[1]{\ensuremath{\textbf{\textit{\uppercase{#1}}}}} 
\newcommand{\kor}[1]{\textcolor{red}{#1}}

\newcommand{\abbpos}{50pt}

%Abkürzungen
% see http://en.wikibooks.org/wiki/LaTeX/Glossary
%new glossary term
\newglossaryentry{sample}{name={sample}, description={a sample entry}}

%new acronym
\newacronym[\glsshortpluralkey=cas,\glslongpluralkey=contrived acronyms]{aca}{aca}{a contrived acronym}%
\newacronym{abk}{Abk.}{Abkürzung}%
\newacronym{http}{HTTP}{hypertext transfer protokol}


% DOKUMENT ANFANG
\begin{document}

%%%% Titelseite Definitionen %%%%

%Header PsyErg
%\def\tabularxcolumn#1{m{#1}}
% \renewcommand{\thetitlehead}{
%\begin{tabularx}{\textwidth}{l X m{6.5cm}}
%\raisebox{-.5\height}{
%\includegraphics[height=1.8cm]{figures/unilogo4cohne.jpg}
%} & &\small{Philosophische Fakultät II\newline
%Institut für Mensch-Computer-Medien\newline{Psychologische Ergonomie}}\\   
%\end{tabularx} }

%Header HCI
\def\tabularxcolumn#1{m{#1}}
 \renewcommand{\thetitlehead}{
\begin{tabularx}{\textwidth}{l X m{6.745cm} r}
\raisebox{-.5\height}{
\includegraphics[height=1.8cm]{figures/unilogo4cohne.jpg}
} & &\small{Fakultät für Mathematik und Informatik\newline
Institut für Informatik\newline{Mensch-Computer Interaktion}}&
\raisebox{-.5\height}{
\includegraphics[height=1.8cm]{figures/hci-logo-red.pdf}
}\\ \end{tabularx} }


%Art der Arbeit (Seminararbeit/Hausarbeit zum Seminar "Seminarname" "Semester", Bachelorarbeit,...)
\renewcommand{\artderarbeit}{ \usekomafont{subject}\textbf{Projektarbeit}  \vspace{0.5cm} \\ \normalfont zum Kurs\\ Grundlagen der Mensch-Computer-Systeme \\ im Studiengang Master Informatik\\ an der Universität Würzburg}

%Titel der Arbeit
\renewcommand{\thetitle}{Cognitive Walkthrough zum Bestellprozess des Onlineshops www.alternate.de}

% ggf. Untertitel der Arbeit
%\renewcommand{\thesubtitle}{ Untertitel}

%Autor(en)
\renewcommand{\theauthor}{vorgelegt von\\ Bastian Hedenkamp - Matrikelnummer: 10101010\\ Michael Gabler - Matrikelnummer: 10101010}

%Datum (Bei Thesis Abgabedatum)
\renewcommand{\thedate}{am \today}

%Betreuer/ Kursleiter
\renewcommand{\betreuer}{Betreuer/Prüfer:\\ Prof. Marc Erich Latoschik, Informatik IX, Universität Würzburg}


%%%% Zusammenfassung %%%%

% obligatorisch bei Abschlussarbeiten
\renewcommand{\theabstractfirst}{deutsche Zusammenfassung - Lorem ipsum dolor sit amet, consetetur sadipscing elitr, sed diam nonumy eirmod tempor invidunt ut labore et dolore magna aliquyam erat, sed diam voluptua. At vero eos et accusam et justo duo dolores et ea rebum. Stet clita kasd gubergren, no sea takimata sanctus est Lorem ipsum dolor sit amet. Lorem ipsum dolor sit amet, consetetur sadipscing elitr, sed diam nonumy eirmod tempor invidunt ut labore et dolore magna aliquyam erat, sed diam voluptua. At vero eos et accusam et justo duo dolores et ea rebum. Stet clita kasd gubergren, no sea takimata sanctus est Lorem ipsum dolor sit amet.   

Duis autem vel eum iriure dolor in hendrerit in vulputate velit esse molestie consequat, vel illum dolore eu feugiat nulla facilisis at vero eros et accumsan et iusto odio dignissim qui blandit praesent luptatum zzril delenit augue duis dolore te feugait nulla facilisi. Lorem ipsum dolor sit amet}

%%%% Titelei %%%%

\frontmatter%bei scrreprt auskommentieren
\begin{spacing}{1}

% Titelseite
\printthetitlepage
\end{spacing}

% Zusammenfassung (obligatorisch bei Abschlussarbeiten)
\printtheabstract{Zusammenfassung} 	%\printtheabstract{titlefirst} nur \abstractfirst mit Titel 'fitlefirst'

\begin{spacing}{1}

% Inhaltsverzeichnis
\tableofcontents 	

% Abbildungsverzeichnis  (optional)
\listoffigures 		

% Tabellenverzeichnis (optional)
\listoftables		

% Abkürzungsverzeichnis
% You have to run:
% makeindex -s "$bfname".ist -t "$bfname".alg -o "$bfname".acr "$bfname".acn
% makeindex -s "$bfname".ist -t "$bfname".glg -o "$bfname".gls "$bfname".glo
\printglossary[type=\acronymtype] 

\end{spacing}

%%%% Inhalte %%%%

\mainmatter %bei scrreprt auskommentieren

\chapter{Einleitung}
Hier kommt die Einleitung\\
Basti

\chapter{Methodik}
Was ist Cognitive Walkthrough?\\
Basti

\chapter{Durchführung}
\section{Vorbereitung}
Personas:\\
- technikinteressiert, mit Onlineshops vertraut, sucht Multimediaartikel\\
- Informatikstudent, sucht Bauteile für PC\\
- Rentner, Hobbybastler oder Hobbygärtner, sucht Gartengeräte, Werkzeug\\
- Netzwerktechniker, sucht beruflich nach Serverzubehör\\
Michi

\section{Analyse}
\subsection{Bestellprozess angemeldet am PC}
Basti

\subsection{Bestellprozess als Gast mit Smartphone}
Michi

\section{Ergebnisse}
Welche Probleme sind bei der Analyse aufgetreten?\\
zwei Absätze -> jeder seine Zusammenfassung

\section{Diskussion}
Wie können die Probleme behoben werden?\\
Basti

\chapter{Prototyp}
Aufbau und Merkmale\\
Michi

\chapter{Zusammenfassung}
Michi

%%%% Anhänge %%%%

\newpage
 \printbibliography	% Literaturvezeichnis (obligatorisch)
%\printbibheading
%\printbibliography[nottype=online,heading=subbibliography,
%                   title={Printed Sources}]
%\printbibliography[type=online,heading=subbibliography,
 %                  title={Online Sources}]

\appendix 				% Anhang (optional) Überschriften beginnen mit Anhang ...


\end{document}
% DOKUMENT ENDEU